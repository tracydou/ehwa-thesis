%%%%%%%%%%%%%%%%%%%%%%%%%%%%%%%%%%%%%%%%%
% Beamer Presentation
% LaTeX Template
% Version 1.0 (10/11/12)
%
% This template has been downloaded from:
% http://www.LaTeXTemplates.com
%
% License:
% CC BY-NC-SA 3.0 (http://creativecommons.org/licenses/by-nc-sa/3.0/)
%
%%%%%%%%%%%%%%%%%%%%%%%%%%%%%%%%%%%%%%%%%

%----------------------------------------------------------------------------------------
%	PACKAGES AND THEMES
%----------------------------------------------------------------------------------------

\documentclass{beamer}

\mode<presentation> {

% The Beamer class comes with a number of default slide themes
% which change the colors and layouts of slides. Below this is a list
% of all the themes, uncomment each in turn to see what they look like.

%\usetheme{default}
%\usetheme{AnnArbor}
%\usetheme{Antibes}
%\usetheme{Bergen}
%\usetheme{Berkeley}
%\usetheme{Berlin}
%\usetheme{Boadilla}
%\usetheme{CambridgeUS}
%\usetheme{Copenhagen}
%\usetheme{Darmstadt}
%\usetheme{Dresden}
%\usetheme{Frankfurt}
%\usetheme{Goettingen}
%\usetheme{Hannover}
%\usetheme{Ilmenau}
%\usetheme{JuanLesPins}
%\usetheme{Luebeck}
\usetheme{Madrid}
%\usetheme{Malmoe}
%\usetheme{Marburg}
%\usetheme{Montpellier}
%\usetheme{PaloAlto}
%\usetheme{Pittsburgh}
%\usetheme{Rochester}
%\usetheme{Singapore}
%\usetheme{Szeged}
%\usetheme{Warsaw}

% As well as themes, the Beamer class has a number of color themes
% for any slide theme. Uncomment each of these in turn to see how it
% changes the colors of your current slide theme.

%\usecolortheme{albatross}
%\usecolortheme{beaver}
%\usecolortheme{beetle}
%\usecolortheme{crane}
%\usecolortheme{dolphin}
%\usecolortheme{dove}
%\usecolortheme{fly}
%\usecolortheme{lily}
%\usecolortheme{orchid}
%\usecolortheme{rose}
%\usecolortheme{seagull}
%\usecolortheme{seahorse}
%\usecolortheme{whale}
%\usecolortheme{wolverine}

%\setbeamertemplate{footline} % To remove the footer line in all slides uncomment this line
%\setbeamertemplate{footline}[page number] % To replace the footer line in all slides with a simple slide count uncomment this line

%\setbeamertemplate{navigation symbols}{} % To remove the navigation symbols from the bottom of all slides uncomment this line
}
\usepackage{graphicx} % Allows including images
\usepackage{booktabs} % Allows the use of \toprule, \midrule and \bottomrule in tables

\usepackage{array}
\newcolumntype{V}[1]{>{\centering\arraybackslash} m{#1} }
\newcolumntype{L}[1]{>{\arraybackslash} m{#1} }

\pdfpageattr {/Group << /S /Transparency /I true /CS /DeviceRGB>>}
%----------------------------------------------------------------------------------------
%	TITLE PAGE
%----------------------------------------------------------------------------------------

\title[Master Thesis Presentation]{An assistive handwashing system with emotional intelligence}
% The short title appears at the bottom of every slide, the full title is only on the title page

\author{Luyuan Lin} % Your name
\institute[UWaterloo] % Your institution as it will appear on the bottom of every slide, may be shorthand to save space
{
University of Waterloo \\ % Your institution for the title page
\medskip
\textit{Supervisor:
\newline Jesse Hoey
} % Your email address
}
\date{\today} % Date, can be changed to a custom date

\begin{document}

\begin{frame}
\titlepage % Print the title page as the first slide
\end{frame}

\begin{frame}
\frametitle{Overview} % Table of contents slide, comment this block out to remove it
\tableofcontents % Throughout your presentation, if you choose to use \section{} and \subsection{} commands, these will automatically be printed on this slide as an overview of your presentation
\end{frame}

%----------------------------------------------------------------------------------------
%	PRESENTATION SLIDES
%----------------------------------------------------------------------------------------

%-----------------------------------------------------------------
\section{Problem Statement} 
% Sections can be created in order to organize your presentation into discrete blocks
% all sections and subsections are automatically printed in the table of contents as an overview of the talk
%------------------------------------------------
\subsection{Motivation}
\begin{frame}
\frametitle{Problem Statement - Motivation}
The COACH system
\begin{itemize}
\item is an assistive system helping with an elder's daily activities
\item monitors a user washing his/her hands
\item detects when the user has lost track of what he/she is doing
\item displays a prerecorded assistive prompt when needed
\item works well for some persons, but not as well for others
\end{itemize}
\pause
\vspace{0.3cm}
Using Emotional Intelligence in Assitive Systems
\begin{itemize}
\pause \item recognization of affective states
\pause \item generation of affective signals
\pause \item study of human emotions
\pause \item computationally modelling affective HCIs
\end{itemize}
\end{frame}

%------------------------------------------------
\subsection{Objectives}
\begin{frame}
\frametitle{Problem Statement - Objectives}
To augment the COACH system with an emotional reasoning engine based on BayesACT so that the augmented system:\\
\begin{itemize}
\pause \item is designed in a portable and extensible way
\pause \item runs in real-time from the perspective of the user group
\pause \item provides at least a level of functional assistance of as high quality as the COACH
\pause \item is able to tune the prompts in some way according to the emotional state of a user
\end{itemize}
\vspace{0.3cm}
\pause Note: The last objective is ill-defined, as the question of how exactly tuning prompts to users will be most effective is not clear at this point.
\end{frame}

%-----------------------------------------------------------------
\section{Basic Concepts}
%------------------------------------------------
\subsection{Affect Control Theory (ACT)}
\begin{frame}
\frametitle{Concepts - ACT}
Affect Control Theory (ACT)
\begin{itemize}
\item represents emotions as vectors that represent evaluation ($E$), potency ($P$), and activity ($A$) respectively
\pause \item describes social events by an Actor-Behaviour-Object (ABO) grammar
\pause \item ``fundamentals'' of identities and behaviours; shared between people within a same culture
\pause \item ``transient impressions'': emotional feelings of people evoked by a specific event
\end{itemize}
\begin{block}{The ACT Principal}
Actors work to experience transient impressions that are consistent with their fundamental sentiments.
\end{block}
\end{frame}

%------------------------------------------------
\subsection{Partially Observable Markov Decision Process (POMDP)}
\begin{frame}
\frametitle{Concepts - POMDP}
Partially Observable Markov Decision Process (POMDP)
\vspace{.3cm}
\begin{columns}[c]
\column{.5\textwidth}
\includegraphics[trim = 15mm 10mm 15mm 10mm, clip, width=\linewidth]{fig/fig-pomdp.pdf}
\column{.5\textwidth}
\begin{itemize}
\item A timeslice of a POMDP process (solid lines)
\pause \item Variables: \{ $X$, $A$, $\mathbf{\Omega_{X}}$ \}
\pause \item $Pr : X \to \Delta(\mathbf{\Omega_{X}})$, $Pr : X \times A \to \Delta(\mathbf{X})$
\pause \item Reward Function: $R(A, X')$
\pause \item Augmented with affective \\ states (dotted lines)
\end{itemize}
\end{columns}
\end{frame}

%------------------------------------------------
\subsection{The BayesACT Framework}
\begin{frame}
\frametitle{Concepts - BayesACT}
\begin{itemize}
\item A Bayesian version of the ACT theory
\item Combines the ACT with POMDP model so that can learn an interactant's identity
\end{itemize}
%insert figure & explain the bayesact framework
\pause
\begin{columns}[c]
\column{.5\textwidth}
\includegraphics[trim = 20mm 10mm 20mm 10mm, clip, width=\linewidth]{fig/fig-bayesact.pdf}
\column{.5\textwidth}
\begin{itemize}
\pause \item States $S = \{F, T, X\}$, where $F = \{F_{ij}\}, T = \{T_{ij}\}, i \in \{a, b, c\}, j \in \{e, p, a\}$
\pause \item Observations $\Omega = \{\Omega_{X}, \Omega_{b}\}$
\pause \item Actions $\{A, B_{a}\}$
\pause \item By updating $F$, the probability distribution of the client's identity $F_{c}$ is learned
\pause \item Calculate $\{A, B_{a}\}$ basing on $\{F, T, X\}$
\end{itemize}
\end{columns}
\end{frame}

\begin{frame}
\frametitle{Concepts - BayesACT cont.}
Updates $F$ and Calculates $\{A, B_{a}\}$ basing on $\{F, T, X\}$
\begin{itemize}
% ---- formula computing deflection ----
\pause \item The deflection $\phi(F, T)$ between $F$ and $T$: 
\begin{equation}\label{eq:eq_deflection}
\phi(f,t) \propto e^{-(f'-t')\Sigma^{-1}(f-t)}
\end{equation}
% ---- formula computing probability of $f'$ ----
\pause \item The probability of a post-action fundamental sentiment $f'$:
\begin{equation}\label{eq:eq_pr_f}
Pr(f'|f,t,x,b_{a},\phi) \propto e^{-\phi(f',t')-\xi(f',f,b_{a},x)} 
\end{equation}
where $t'$ can be computed from $\{f', t, x\}$ by empirically derived prediction equations of ACT.
% ---- how the application progresses (i.e. how planstep changes in our case) ----
\pause \item $Pr(x'|x,f',t',a)$: how the application progresses
% ---- observation functions ----
\pause \item $Pr(\omega_{b}|f)$ and $Pr(\omega_{x}|x)$: observation functions for the client behaviour sentiment and system state 
\end{itemize}
\end{frame}

%-----------------------------------------------------------------
\section{Solution: System Design and Implementation}
%------------------------------------------------
\begin{frame}
\frametitle{Solution - Overview}
%insert figure describing the functionality of each component of the system
%state how each component is related to our objective
\end{frame}

%------------------------------------------------
\subsection{Components}
\begin{frame}
\frametitle{Solution - the Planstep and Emotion Updater}
%insert figure: the model & how emotional states are updated
%note: there's a noise parameter describing the confidence of a certain result
\end{frame}

\begin{frame}
\frametitle{Solution - the Planstep and Emotion Updater cont.}
%pseudo-code: how planstep is updated according to behavioural obs, awareness, & deflection
\end{frame}

\begin{frame}
\frametitle{Solution - the Planstep and Emotion Updater cont.}
%state the server-client model is used & a "buffer" component is involved
\end{frame}

\begin{frame}
\frametitle{Solution - the EPA-Calculator}
%feature selection (other feature-selection approaches are not feasible in current situation)
%describe the threshold-based method (i.e. piecewise linear interpolation)
\end{frame}

\begin{frame}
\frametitle{Solution - the Observer}
%state how the original hand-tracker works
%state how the server-client model is used
\end{frame}

\begin{frame}
\frametitle{Solution - the Output Part}
%describe the survey - play two videos here: examples of the prompts
%describe how to select (include formula defining the distance of prompts)
\end{frame}

%------------------------------------------------
\subsection{Coordination between components}
\begin{frame}
\frametitle{Solution - the Buffer}
%the position: between the EPA-Calc & Observer and the Reasoning Engine
%functionality of the buffer: control timings & smoothing values
\end{frame}

%-----------------------------------------------------------------
\section{Experimental Results}
%-----------------------------------------------------------
\begin{frame}
\frametitle{Experiments - Variables and Parameters}
%planstep definitions: the eight ps & 5 behaviours monitored
%locations of "certain objects": related to the 5 behaviours (give a screenshot of the physical settings; or mention would give)
  * table defining all the parameters - explain the meanings of all the parameters
\end{frame}

\begin{frame}
\frametitle{Experiments - Variables and Parameters cont.}
%table defining all the parameters - explain the meanings of all the parameters
\end{frame}

\begin{frame}
\frametitle{Experiments - Test \#1}
%table 1 + video 1: what happened in run 1
%explanation focus on planstep responses
%mention emotional responses in general
\end{frame}

\begin{frame}
\frametitle{Experiments - Test \#1 cont.}
%(if needed) table 1 + video 1: what happened in run 1
\end{frame}

\begin{frame}
\frametitle{Experiments - Test \#2}
%video 2 + table 2: what happened in run 2
%explanation focus on planstep responses
%mention emotional responses in general
\end{frame}

\begin{frame}
\frametitle{Experiments - Test \#2 cont.}
%(if needed) table 2: what happened in run 2
\end{frame}

\begin{frame}
\frametitle{Experiments - Conclusion}
%planstep updater - they can update accordingly
%emotion updater - insert table & state the correlations found between EPA values
%mention that more experiments were run & result is in the appendix 
\end{frame}

%-----------------------------------------------------------------
\section{Discussion}
%-----------------------------------------------------------

\subsection{Contribution}
\begin{frame}
\frametitle{Discussion - Contribution}
%review the four objectives
%state contribution of this paper
\end{frame}

\subsection{Future Work}
\begin{frame}
\frametitle{Discussion - Future Work}

\end{frame}

%---------------------------------------------------------------
% Ending pages
%---------------------------------------------------------------
\begin{frame}
\frametitle{References}
[1] The bayesact paper
\newline [2] The tracker paper.
\newline [3] The survey paper.
\end{frame}
%------------------------------------------------
\begin{frame}
\frametitle{Acknowledgement}
Jesse Hoey
\newline James Tung and Peter van Beek
\newline Xiao Yang, Chengbo Li and Enxun Wei
\end{frame}
%------------------------------------------------
\begin{frame}
\frametitle{The end}
\Huge{\centerline{Thank you!}}   
\fontsize{5mm}{4mm}
\begin{itemize}
\item Questions?
\item Comments?
\end{itemize}
\end{frame}
%----------------------------------------------------------------------------------------

%---------------------------------------------------------------
% Additional pages
% put pages in assistance to answering questions here
%---------------------------------------------------------------

\end{document} 
