\chapter{Introduction}
\label{chap:intro}

According to the United Nation's population report 2013 \footnote{\url{http://www.un.org/en/development/desa/population/publications/pdf/ageing/WorldPopulationAgeing2013.pdf}}, we are experiencing an aging world. The global share of older people (aged 60 years or over) increased from 9.2\% in 1990 to 11.7\%  in 2013 and will continue to grow as a proportion of the world population, reaching 21.1\% by 2050. Currently around the world, there are about 40\% of older persons aged 60 years or over who live independently, that is to say, alone or with their spouse only. As countries develop and their populations continue to age, living alone or with a spouse only will likely become much more common among older people in the future. While many elders remain healthy and productive, overall this segment of the population is subject to physical and cognitive impairment at higher rates than younger people. Take a broad category of brain diseases, dementia, for example. While only 3\% of people between the ages of 65 to 74 have dementia, 47\% of people over the age of 85 have some form of dementia \cite{budson2011memory}. As more people are living longer, dementia is becoming more common. Dementia can cause long term loss of ability to think and reason clearly. Persons with dementia (e.g. Alzheimer's disease) are reported to have difficulty in daily functioning, such as hand-washing, preparing food and dressing. For example, while performing a task in daily life, persons with Alzheimer's disease (AD) may forget how much of the task he or she has completed, what an object looks like, or what the necessary steps are.

Luckily, more and more new technologies that incorporate Artificial Intelligence (AI) methods have been explored to build assistive systems that can help with an elder's everyday lives. Smart home systems are being developed to help older adults with AD in a variety of ways, for example, in automated reminders for tasks like handwashing \cite{hoey2010automated} and meal preparing \cite{philipose2004inferring}, and providing social and cognitive stimulation \cite{kaye2011intelligent}. The assistive systems designed for the many and varied difficulties faced by older adults with cognitive disabilities typically take the form of automated methods of monitoring the users' behaviours \cite{hoey2012lacasa}, assessing the users' cognitive levels, and assisting the users to complete daily tasks by providing prompts when necessary \cite{boger2005decision, peters2014automatic, hoey2010automated}. However, even when the solutions satisfy functional requirements, the systems' effectiveness may still be limited by a lack of an affective (emotional) connection between the systems and their users.

A COACH system has been developed to assist older adults with dementia to carry out basic daily activities (e.g. hand-washing) \cite{boger2005decision, mihailidis2008coach}. The system is effective at monitoring a user washing his/her hands, detecting when the user has lost track of what he/she is doing, and when needed, displaying a prerecorded assistive prompt \cite{mihailidis2008coach}. However, while the system works well for some persons, it does not perform as well for others. One primary reason for this may be the limitation of the pre-recorded prompts in capturing the heterogeneity in socio-cultural and personal affective identities of the users. Each user of the system comes from a different background and has a different sense of ``self''. Thus, the users would have different emotional responses to the prompts given. For example, one person may find the prompt helpful and motivational, while another may find it imperious and impatient, and prefer to a more servile instructional message. The disapprovement feelings caused by the prompts may affect a user's responsiveness. Apparently, the ``one size fits all'' style of prompting is not enough for an assistive system aiming at high cross-user performance. More sophisticated methods of generating prompts that align with the user's affective states (such as identities) are needed.

Affect Control Theory (ACT) \cite{robinson2006affect} is a well established sociological theory that models affective reasoning during human interactions. It represents all affective meanings, such as those of identities and behaviours, by three-dimensional ``EPA'' vectors: evaluation (E, e.g. how positive), potency (P, e.g. how powerful), and activity (A, e.g. how active). The theory hypothesizes that people have fundamental sentiments about their identities and would act to minimize the deflection of transient impressions caused by social events from the fundamentals. ACT hypothesizes that the fundamental sentiments of identities and behaviours are shared within a same culture. These hypotheses has been supported by a variety of studies. Tests of ACT's validity on both verbal (e.g. INTERACT\footnote{Program accessible via \url{http://www.indiana.edu/~socpsy/ACT/interact.htm}. Readers are refered to \url{http://www.indiana.edu/~socpsy/ACT/references.html} for more readings on ACT.}) and non-verbal behaviours \cite{schroder2013culture} have been reported. Based on ACT, an interactant's behaviours can be predicted given his/her affective state (i.e. his/her affective identity\footnote{In this thesis, the term \textit{identity} is used to denote a kind of person in a social situation.} and how he/she perceives the situation emotionally).

Identities of persons with AD are not easy to obtain. Studies have shown that the identities of persons with dementia are changed by the disease \cite{orona1990temporality}, and that persons with AD have more vague or abstract notions of their identity \cite{rose2004memory}. To tackle this problem, a BayesACT model to learn interactants' identities has been formulated on the basis of the original ACT \cite{hoey2013bayesian}. Based on the BayesACT model, the authors of \cite{hoey2013bayesian} have built a program, in simulation, can assist persons with AD to complete the handwashing task. Though the program is able to provide functional prompts while reasoning about the users' identities and emotions, it consumes pre-processed observation information (including user behaviour labels and the affective meaning as EPA vectors of the behaviours) as input, as opposed to perceiving the information from the environment itself. The program is not able to construct and display real prompts to its users as well; instead, it represents the prompts to be displayed by functional and emotional labels, with the former one describing the instructional content that should be contained in the prompts and the latter one indicating the affective meanings  (again, EPA vectors) desired in these prompts.

In this thesis, based on the COACH and the BayesACT approaches, we developed a prototype of an assistive system that monitors a person with AD during a handwashing process, learns about the affective identity of the person, and provides prompts that both instruct what the person should perform next to complete the handwashing task, and simultaneously correspond with the person's emotional states. The system uses an RGB-D camera to track the user's hands while hand-washing, and recognizes functional meanings (e.g. has the person turned the water on?) and affective meanings (e.g. is the person active and feeling powerful?) of the user's behaviours. The detected functional and affective meanings of the user behaviours are then fed into a reasoning engine where the system's belief states, including how much the user has completed in the handwashing task and what the the user's affective identity is, are updated. The reasoning engine uses a partially observable Markov decision process (POMDP) to update the system's belief states, and the affective component is based on ACT. The POMDP policy of the reasoning engine then produces an approximately optimal action for the system to take, with the actions described with both functional (e.g. the instructional content of the prompt) and affective (e.g. how the content should be expressed) meanings. A most appropriate audio-visual prompt is finally selected from a set of pre-generated prompts with both functional labels and affective ratings. The final prompt is chosen in a way that both of its functional label and emotional rating are consistent with the functional and emotional meanings of the desired prompt recommended by the reasoning engine.

The goal of this thesis is to show that it is functionally possible to integrate the emotional reasoning of ACT with an existing cognitive assistive technology, the COACH. It focuses on the integration work of fitting emotional intelligence in a functional system, while pointing out directions for future improvements. The contribution of this thesis is therefore to demonstrate, in a controlled laboratory setting with human actors only, how emotional reasoning, a key missing component of most assistive systems, can be integrated into a cognitive intelligent assistive system. The objectives of this thesis is as following:

\textbf{\textit{Objectives.}} To augment the COACH system with an emotional reasoning engine based on BayesACT so that the augmented system: (1) is designed in a portable and extensible way; (2) runs in real-time for the perspective of the user group; (3) provides at least a level of functional assistance of as high quality as the COACH; (4) is able to tune the prompts in some way according to the emotional state of a user. The last objective (4) is ill-defined, as the question of how exactly tuning prompts to users will be most effective is not clear at this point.

The thesis is structured as follows. Basic concepts used throughout the thesis are defined in Chapter~\ref{chap:bg}. Related previous studies, including approaches in building high performance assistive systems, are reviewed in this chapter as well. With the importance of including emotional intelligence in the design of human-computer interaction (HCI) discussed, Chapter~\ref{chap:bg} then briefly examines previous works in the topic of emotional intelligence, i.e. previous research in the areas of \textit{recognition of affective states, generation of affectively modulated signals, psychological study of human emotions, and computationally modeling affective HCIs.} Chapter~\ref{chap:design} of the thesis discusses the challenges in designing an assistive handwashing system that has combined all the aforementioned aspects of emotional intelligence altogether. The whole system is divided into independent components, and both general analysis and detailed examinations of input and output requirements and design difficulties of each of the components are provided. Communication between the components is discussed in the chapter as well. Finally, Chapter~\ref{chap:design} explains the design of the system as an integration of independent components, among which some components are designed as extensions to existing programs. Chapter~\ref{chap:impl} describes how the system is implemented in detail. Preliminary experimental results are presented in Chapter~\ref{chap:impl} as well. Chapter~\ref{chap:discuss} wraps up the thesis by discussing the contributions of this thesis and possible future works.
